	\newsheet{Day 3: Higher Dimensions}{3}{3}


Recall our guiding question:
\begin{center}
	\textbf{Given a basis, what are the shortest vectors}
	
	\textbf{in the lattice generated by this basis?}
\end{center}

We have an algorithm to solve this question in $2$ dimensions, using the notion of a ``reduced basis.'' But as soon as we increase the dimension, we can have bases that are reduced (no basis vector can be shortened by sliding it along the direction of another basis vector) but are still hiding short vectors in the lattice they generate, for example:
	\[\b_1=\begin{pmatrix}
	1\\0\\-10^{-6}
	\end{pmatrix},\,\b_2=\begin{pmatrix}
	1/2 \\ \sqrt{3}/2 \\ 2\cdot 10^{-6}
	\end{pmatrix},\,\b_3=\begin{pmatrix}
	-1/2 \\ \sqrt{3}/2 \\ -3\cdot 10^{-6}
	\end{pmatrix},\,\b_2-\b_1-\b_3=\begin{pmatrix}
	0 \\ 0 \\ 6\cdot 10^{-6}
	\end{pmatrix}.\]

\begin{defns}

A set of vectors $\b_1,\ldots,\b_n\in\R^n$ is a \textbf{\color{magenta}basis} if any vector in $\R^n$ can be written $r_1\b_1+\cdots+r_n\b_n$ for exactly one choice of real numbers $r_1,\ldots,r_n$.
\\

Given a {\color{magenta}basis $\{\b_1,\ldots,\b_n\}$}, the \textbf{\color{gray} lattice generated by this basis} is the set
	\[\LL=\{a_1\b_1+a_2\b_2+\cdots+a_n\b_n\mid a_1,a_2,\ldots,a_n\in \Z\}.\]

The {\color{magenta}fundamental parallelepiped of the basis} is the set
	\[P=\{r_1\b_1+r_2\b_2+\cdots+r_n\b_n\mid r_1,r_2,\ldots,r_n\in [0,1)\}.\]
\end{defns}

\begin{comment}
A basis matrix will always be invertible\footnote{A matrix $M$ is invertible if there exists a matrix $M^{-1}$ with the property that $MM^{-1}$ is the identity matrix ($1$s on the top-left-to-bottom-right diagonal, and $0$s everywhere else).} (Exploration~\ref{exp:basisinv}).

\begin{explor}\label{exp:basisinv}
	Become convinced that any basis matrix is, in fact, invertible. \color{DarkGreen}(Hint: Think of the matrix as a linear map. What does it do to vectors in the standard basis?)
\end{explor}
\end{comment}

\section*{Minkowski's Theorem}

Let's work towards a proof of the following result. A subset $S$ of $\R^n$ is \emph{symmetric} if whenever $\v\in S$, then also $-\v\in S$. The set $S$ is \emph{convex} if whenever $\v,\w\in S$, all vectors on the line segment connecting $\v$ to $\w$ are also in $S$.

\begin{toprove}[(Minkowski's Theorem)]{Theorem}\label{thm:minkowski}
	Let $S$ be a convex, symmetrix subset of $\R^n$, and let $P$ be the fundamental parallelepiped of a basis generating some lattice $\LL$. If $\text{Vol}(S)>2^n\text{Vol}(P)$, then $S$ contains a nonzero element of $\LL$.
\end{toprove}

\begin{explor}
	We can define a function $\R^n\to P$ as follows: if a vector can be written as $\l+\p$ for $\l\in\LL$ and $\p\in P$, send $\l+\p$ to $\p$. Describe this function geometrically.
\end{explor}

\begin{explor}
	Let $S/2$ denote the set $S$ shrunk down by a factor of $2$. If $\text{Vol}(S/2)>\text{Vol}(P)$, conclude that there must be two vectors in $S/2$ which differ by a nonzero lattice vector $\l$.
\end{explor}

\begin{explor}
	Use properties of $S$ to show that $\frac12\l\in S/2$. Then scale back up and finish the proof. 
\end{explor}

How does Minkowski's Theorem help us with the short vector question? The key insight is that we can turn a \emph{length} question (is there a vector with length at most $d$) into a \emph{volume} question (is there a vector in the ball of radius $d$). 

\begin{explor}
	Prove that the $n$-dimensional ball of radius $\sqrt{n}$ has volume greater than $2^n$. Use this to prove Corollary~\ref{cor:lengthbound}.
\end{explor}

\begin{toprove}[\color{white}(of Theorem~{\hypersetup{linkcolor=white}\ref{thm:minkowski}})]{Corollary}\label{cor:lengthbound}
	Let $P$ be a fundamental parallelepiped for $\LL$. Then there exists a nonzero vector $\l\in\LL$ with
	\[\|\l\|\leq \sqrt{n}\text{Vol}(P)^{1/n}.\]
\end{toprove}

\begin{explor}
	Consider the basis discussed at the beginning of this sheet. The parallelepiped it generates has volume $3\sqrt{3}/10^6$.\footnote{If you want to know how to compute this, see Fact~\ref{fact:detvol} in the Optional Exploration section.} Does Corollary~\ref{cor:lengthbound} predict the existence of the short vector we found? How far off is the bound?
\end{explor}

\section*{Successive Minima and Minkowski's Second Theorem}

Minkowski's Theorem is an upper bound: it guarantees the existence of a short vector in our lattice. A lower bound would be nice too (to tell us that the shortest vector can't be \emph{toooo} short), but unfortunately the volume of the fundamental parallelepiped is not enough to give us such a result:
\begin{explor}
	Find a basis in $\R^2$ such that the area of the fundamental parallelogram is $1$, but the length of the shortest vector is very very very very small.
\end{explor}
The issue is that you can make one vector extremely small without changing the volume of the fundamental parallelepiped, as long as you compensate by making other vectors longer.

\begin{defn}
The \textbf{succesive minima} of a lattice, $\lambda_1\leq\cdots\leq\lambda_n$, are defined by the following property: $\lambda_i$ is the smallest number such that there exist at least $i$ \emph{linearly independent} vectors of length at most $\lambda_i$. 
\end{defn}
If you picture a ball with growing radius, then $\lambda_i$ represents the moment that the ball swallows up a vector pointing in an $i^\text{th}$ new direction. In particular, $\lambda_1$ is the length of the shortest nonzero vector in the lattice. So Minkowski's (first) theorem tells us that if $B$ is an $n$-dimensional ball of radius $1$, then
\[\lambda_1^n\text{Vol}(B)\leq 2^n\text{Vol}(P).\]
Minkowski's Second Theorem is similar, except that it uses all the successive minima.
\begin{toprove}[(Minkowski's Second Theorem)]{Theorem}\label{thm:minkowski2}
	Let $B$ be an $n$-dimensional ball of radius $1$, $P$ be the fundamental parallelepiped of a lattice $\LL$, and $\lambda_1,\ldots,\lambda_n$ the successive minima of $\LL$. Then
	\[\frac{1}{n!}2^n\text{Vol}(P)\leq \lambda_1\cdots\lambda_n\text{Vol}(B)\leq 2^n\text{Vol}(P).\]
\end{toprove}

This gives us quite a bit of additional information about the ``shape'' of our lattice.

\begin{explor}
	Looking again at the basis $\{\b_1,\b_2,\b_3\}$ from the beginning of this sheet, we found a very short vector $\x$. Prove that there are no other ``surprise'' short vectors in the lattice. Specifically, any vector that is not a multiple of $\x$ must have length at least $0.275$.
\end{explor}


\section*{Optional Exploration: The Lower Bound}
The upper bound in Theorem~\ref{thm:minkowski2} is rather deep and involved, but we can prove the lower bound! Let $\x_1,\ldots,\x_n$ be vectors in a lattice $\LL$ such that $\x_i$ has length $\lambda_i$.


\begin{explor}
	We took $\x_1,\ldots,\x_n$ to be a set of linearly independent vectors that are as short as possible, so you might expect them to form a basis for $\LL$. This is true in $2$ dimensions: if $\LL$ is generated by a reduced basis $\{\u,\v\}$, then $\lambda_1=\|\u\|$ and $\lambda_2=\|\v\|$. 
	
	But consider the lattice $\LL$ in $\R^5$ generated by
	\[\begin{psmallmatrix}
	1\\0\\0\\0\\0
	\end{psmallmatrix},\,\begin{psmallmatrix}
	0\\1\\0\\0\\0
	\end{psmallmatrix},\,\begin{psmallmatrix}
	0\\0\\1\\0\\0
	\end{psmallmatrix},\,\begin{psmallmatrix}
	0\\0\\0\\1\\0
	\end{psmallmatrix},\,\begin{psmallmatrix}
	1/2\\1/2\\1/2\\1/2\\1/2
	\end{psmallmatrix}.\]
	Show that $\lambda_1=\cdots=\lambda_5=1$, but any basis for $\LL$ must contain a vector of length at least $\sqrt{5}/2>1$. You can't get a basis just by taking the shortest linearly independent vectors! Low-dimensional intuition fails us here; high-dimensional lattices are \emph{weird}.
\end{explor}

\begin{toprove}{Fact}\label{fact:detvol}
	The parallelepiped generated by $\v_1,\ldots,\v_n$,
	\[\{r_1\v_1+r_2\v_2+\cdots+r_n\v_n\mid r_1,r_2,\ldots,r_n\in [0,1]\},\]
	has volume equal to $|\det V|$, where
	\[V=\begin{pmatrix}
	\mid & \mid & & \mid\\
	\v_1 & \v_2 & \cdots & \v_n\\
	\mid & \mid & & \mid	
	\end{pmatrix}=\begin{pmatrix}
	v_{11} & v_{21} & \cdots & v_{n1}\\
	v_{12} & v_{21} & \cdots & v_{n2}\\
	\vdots & \vdots & \ddots & \vdots\\
	v_{1n} & v_{2n} & \cdots & v_{nn}
	\end{pmatrix}\]
	has each vector $\v_i$ as a column.
\end{toprove}
\noindent
See \url{https://textbooks.math.gatech.edu/ila/determinants-volumes.html} for a proof.

\begin{explor}
	If $\b_1,\ldots,\b_n$ is a basis for $\LL$, show that we can write 
	\[\begin{pmatrix}
	\mid & \mid & & \mid\\
	\x_1 & \x_2 & \cdots & \x_n\\
	\mid & \mid & & \mid	
	\end{pmatrix}=\begin{pmatrix}
	\mid & \mid & & \mid\\
	\b_1 & \b_2 & \cdots & \b_n\\
	\mid & \mid & & \mid	
	\end{pmatrix}C\]
	for some matrix $C$ with integer entries. Letting $P'$ denote the parallelepiped generated by $\x_1,\ldots,\x_n$, show that $\text{Vol}(P')\geq \text{Vol}(P)$. 
\end{explor}

\begin{explor}
	Consider the region $T$ in $\R^n$ bounded by vertices at $\pm \x_i$ for all $i$. Compute the volume of $T$ in terms of the volume of $P'$. \color{DarkGreen}(As a warm-up, compare the volume of the region bounded by the vertices $\zero,\x_1,\x_2,\ldots,\x_n$ to the volume of the parallelepiped generated by $\x_1,\x_2,\ldots,\x_n$. Starting with $2$ or $3$ dimensions might help.)
\end{explor}

\begin{explor}
	Consider the region $T'$ in $\R^n$ bounded by vertices at $\pm \x_i/\lambda_i$ for all $i$. Show that $T'$ is contained in $B$. How is the volume of $T'$ related to the volume of $T$?
\end{explor}