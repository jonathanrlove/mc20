\usepackage[utf8]{inputenc}
\usepackage{amsmath}
\usepackage{amsfonts}
\usepackage{amssymb}
\usepackage{ mathrsfs }
\usepackage[svgnames]{xcolor}
\usepackage[pdftex]{graphicx}
\usepackage{amsthm}
\usepackage[hmargin=25mm,vmargin=30mm]{geometry}
\usepackage{tikz}
\usetikzlibrary{decorations.markings}
\usetikzlibrary{arrows.meta}
%\usepackage{tikz-cd}
\usepackage{mathtools}
\usepackage{verbatim}
\usepackage{enumitem}
\usepackage{fancyhdr}
\usepackage{ifthen}
\usepackage{endnotes}
\usepackage{tcolorbox}
\usepackage{csquotes}
\tcbuselibrary{skins,breakable}
\usetikzlibrary{shadings,shadows}
\pagestyle{fancy}

% Class Data variables (for headers)
\def\classname{}
\def\instructors{}
\def\year{}
\def\weeknum{}
\def\daytitle{}

% Use before starting document
\newcommand{\classdata}[3]{
    \def\classname{#1}
    \def\instructors{#2}
    \def\year{#3}
}

\newcounter{blockcounter}

% Use to start new sheets
\newcommand{\newsheet}[3]{
    \pagebreak
    \noindent
    \begin{center}
        \LARGE #1   
    \end{center}
    \vspace{4mm}
    
    \thispagestyle{plain}
    \def\weeknum{#2}
    \def\daytitle{#1}
    \setcounter{page}{1}
    \setcounter{section}{#3}
    \setcounter{exercise-thm}{0}
    \setcounter{blockcounter}{0}
    \setcounter{lem}{0}
}

% Standard page headers
\fancyhead[L]{\classname}
\fancyhead[C]{}
\fancyhead[R]{\daytitle}
\fancyfoot[L,R]{}
\fancyfoot[C]{\thepage}
\renewcommand{\headrulewidth}{0.4pt}

% headers for first pages of new sections
\fancypagestyle{plain}{ 
\fancyhf{}
\fancyhead[L]{\classname}
\fancyhead[C]{\textit{Mathcamp \year, Week \weeknum}}
\fancyhead[R]{\instructors}
\fancyfoot[L,R]{}
\fancyfoot[C]{\thepage}
\renewcommand{\headrulewidth}{0.4pt}
\renewcommand{\footrulewidth}{0pt}}


\usepackage{hyperref}
\hypersetup{
    colorlinks,
    citecolor=black,
    filecolor=black,
    linkcolor=black,
    urlcolor=black
}



\newtheorem{lem}{Lemma}[section]
\newtheorem{thm}[lem]{Theorem}
\newtheorem{fact}[lem]{Fact}
\newtheorem{prop}[lem]{Proposition}
\newtheorem{cor}[lem]{Corollary}
\newtheorem{observation}[lem]{Observation}
\newtheorem{goal}[lem]{Goal}

\theoremstyle{definition}
\newtheorem{rem}[lem]{Remark}
\newtheorem{example}[lem]{Example}
\newtheorem{exmps}[lem]{Examples}
\newtheorem{postulate}[lem]{Postulate}
\newtheorem{exercise-thm}{Exercise}[section]
\newtheorem{explor-thm}[exercise-thm]{Exploration}

\def\showanswers{false}
\newcommand{\answer}[1]
{
	\ifthenelse{\equal{\showanswers}{true}}{\textcolor{gray}{Solution: #1}}{}
}

\newlength{\boxsep}
\setlength{\boxsep}{10pt}

\newenvironment{explor}[1][]{
	\refstepcounter{blockcounter}
	\vspace{\boxsep}
	\tcolorbox[
	noparskip,
	colback=LightGreen,colframe=DarkGreen,%
	colbacklower=LimeGreen!75!LightGreen,%
	title=\textbf{Exploration \theblockcounter\ #1}]
	}{
	\endtcolorbox
	\vspace{\boxsep}
}

\newenvironment{toprove}[2][]{
	\refstepcounter{blockcounter}
	\vspace{\boxsep}
	\tcolorbox[
	noparskip,
	colback=LightBlue,colframe=DarkBlue,%
	colbacklower=DarkBlue!75!LightBlue,%
	title=\textbf{#2 \theblockcounter\ #1}]
}{
	\endtcolorbox
	\vspace{\boxsep}
}

\newenvironment{reference}{
	\vspace{\boxsep}
	\tcolorbox[
	noparskip,
	colback=LightGray,colframe=DarkGray,%
	colbacklower=DarkGray!75!LightGray,%
	title=\textbf{Reference}]
}{
	\endtcolorbox
	\vspace{\boxsep}
}
\newenvironment{defn}{
	\vspace{\boxsep}
	\tcolorbox[
	noparskip,
	colback=LightGray,colframe=DarkGray,%
	colbacklower=DarkGray!75!LightGray,%
	title=\textbf{Definition}]
}{
	\endtcolorbox
	\vspace{\boxsep}
}
\newenvironment{defns}{
	\vspace{\boxsep}
	\tcolorbox[
	noparskip,
	colback=LightGray,colframe=DarkGray,%
	colbacklower=DarkGray!75!LightGray,%
	title=\textbf{Definitions}]
}{
	\endtcolorbox
	\vspace{\boxsep}
}



\newcommand\nc{\newcommand}
\newcommand{\mute}[1] {}
\nc{\ZZ}{\mathbb Z}
\nc{\FF}{\mathbb F}
\nc{\RR}{\mathbb R}
\nc{\QQ}{\mathbb Q}
\nc{\NN}{\mathbb N}
\nc{\HH}{\mathbb H}
\nc{\CC}{\mathbb C}
\nc{\OO}{\mathbb O}
\DeclareMathOperator\Hi{\mathsf{Hi}}

\def \N{\mathbb{N}}
\def \Z{\mathbb{Z}}
\def \R{\mathbb{R}}
\def \C{\mathbb{C}}
\def \Q{\mathbb{Q}}
\def \Re{\mathrm{Re}}
\def \Im{\mathrm{Im}}
\def \one{\mathbf{1}}

\def \vphi{\varphi}
\def \eps{\varepsilon}
\def \ssm{\smallsetminus}
\def \u{\mathbf{u}}
\def \v{\mathbf{v}}
\def \w{\mathbf{w}}
\def \p{\mathbf{p}}
\def \q{\mathbf{q}}
\def \l{\mathbf{l}}
\def \x{\mathbf{x}}
\def \b{\mathbf{b}}
\def \zero{\mathbf{0}}
\DeclareMathOperator\GL{GL}

\def \LL{\mathcal{L}}

\usepackage{relsize}
\newcommand*{\LargerCdot}{\raisebox{-0.25ex}{\scalebox{1.4}{$\cdot$}}}



\tikzset{->-/.style={decoration={
			markings,
			mark=at position .7 with {\arrow{>[scale=5,
					length=2,
					width=2]}}},postaction={decorate}}}